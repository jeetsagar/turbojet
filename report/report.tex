\documentclass[a4paper,12pt]{article}

%-------------------------------------------------------------------------------

\usepackage[final]{nips_2017}
\usepackage[utf8]{inputenc}

\usepackage{hyperref}
\usepackage{url}

\usepackage{amsmath}
\usepackage{amsthm}
\usepackage{amssymb}
\usepackage{amsfonts}

\usepackage{microtype}

\usepackage{graphicx}
\usepackage{tikz}

% \usepackage{natbib}

%-------------------------------------------------------------------------------

% \hypersetup{
%     colorlinks=true,
%     linkcolor=red,
%     filecolor=magenta,
%     urlcolor=cyan
% }

\DeclareMathOperator{\rmse}{RMSE}
\DeclareMathOperator{\nasa}{S}

%-------------------------------------------------------------------------------

\title{Prognostic Health Management for Turbofan Engines Using Deep Learning}

\author{
Aditya Gulati\\
\texttt{adgulati@stanford.edu} \\
\And
Jeetsagar Ghorai \\
\texttt{jghorai@stanford.edu} \\
}

%-------------------------------------------------------------------------------
\begin{document}

\begin{center}
\includegraphics[width=3cm, height=0.7cm]{CS230}
\end{center}
\maketitle

%-------------------------------------------------------------------------------

\section{Introduction}

Prognostic Health Management (\emph{PHM}) is a unified framework for forecasting
system health and reliability. Most systems of interest are composed of multiple
components. Failure of a component in a system can result in adverse outcomes
such as stoppage of operation, destruction of the system or loss of life. In
most cases, the failure of a component results from the degradation of said
component over the course of operation. Prognostic Health Management is
concerned with forecasting potential failures of systems by monitoring the
status of the components and the performance of the system.\(^{[1]}\) PHM is an
active area of research in reliability engineering and PHM techniques have been
applied to a variety of systems such as hydraulic pumps, Lithium ion batteries,
MOSFETs etc.

In most problems of interest, the data is available in the form of a time series
of sensor readings. Given this time series data, the aim of PHM is to predict
the Remaining Useful Life (\emph{RUL}) of the system. In industrial systems,
failure of a single critical component can halt an entire production line until
the component is replaced. This can result in lost productivity and can disrupt
supply chains upstream. Predicting the potential failure of a component allows
the operator to plan for repair or replacement, mitigate downtime and ensure the
safety of the equipment, the environment and workers. Overestimating RUL leads
to an unplanned failure, whereas underestimating RUL leads to underutilization
of the component. In safety critical applications, overestimation poses a higher
risk. However, severe underestimation is also undesriable.

%-------------------------------------------------------------------------------

\section{Dataset}

The aim of our work is to develop a prognostics model for turbofan aircraft
engines using deep learning. We have used the
\href{https://ti.arc.nasa.gov/tech/dash/groups/pcoe/prognostic-data-repository}
{Turbofan Engine Degradation Simulation Data Set-2} published by the Prognostics
Center of Excellence at NASA. This dataset contains run-to-failure trajectories
of a number of turbofan aircraft engines.\(^{[3]}\)

The published repository contains multiple datasets. One representative dataset,
\texttt{DS02}, consists of run-to-failure simulation data for nine engines. In
this dataset, the operating conditions are described using 4 attributes. The
model outputs the values of 14 measured physical properties and the readings
from 14 virtual sensors. Together, there are 32 features at every time-step. In
the dataset, the different engines are referred to as units. The units with
$u = 2, 5, 10, 16, 18, 20$ are the six training units and the units with
$u = 11, 14, 15$ are the three test units. Some characteristics of the dataset
are shown in Figure~\ref{fig:flight_profile} and Figure~\ref{fig:unit_kde}.

\begin{figure}
    \centering
    \includegraphics[width=\linewidth]{flight_profile.png}
    \caption{How the altitude, Mach number, throttle-resolver angle and temperature at fan inlet changes throughout a single flight of unit 20}
    \label{fig:flight_profile}
\end{figure}


\begin{figure}
    \centering
    \includegraphics[width=\linewidth]{kde.png}
    \caption{Kernel density estimations of altitude, Mach number, throttle resolver angle and temperature at fan inlet for 6 training and 3 test units. The flight characteristics of units 14 and 15 are different from those of the training units.}
    \label{fig:unit_kde}
\end{figure}

%-------------------------------------------------------------------------------

\section{Problem Statement}

The dataset consists of the run-to-failure trajectories of $M+N$ units. There
are $s$ parameters which denote the operating condition of the aircraft and the
output of $p$ sensors are monitored at each time-step.

Suppose that the unit $i$ was operated for $m_i$ time-steps. The sensor
readings at time-step $t$ is denoted by $x_{i}^{(t)} \in \mathbb{R}^{p}$ and the
operating condition at time-step $t$ is denoted by
$w_{i}^{(t)} \in \mathbb{R}^{s}$. The RUL at time-step $t$ is denoted by
$y_{i}^{t} \in \mathbb{R}$.

Suppose that $X_i = \lbrack x_{i}^{(1)}, \ldots, x_{i}^{(m_i)} \rbrack^{T}$,
$W_i = \lbrack w_{i}^{(1)}, \ldots, w_{i}^{(m_i)} \rbrack^{T}$ and
$Y_i = \lbrack y_{i}^{1}, \ldots, y_{i}^{m_i} \rbrack^{T}$ Then,
$\mathcal{D}_{train} = \{W_i, X_i, Y_i\}_{i=1}^{N}$ constitutes the training
dataset and $\mathcal{D}_{test} = \{W_i, X_i, Y_i\}_{i=1}^{M}$ constitutes the
test dataset.

The aim of our project is to use $\mathcal{D}_{train}$ to train a deep learning
model $\mathcal{G}$ that predicts $\hat{Y}$, the RUL of the units in the test
dataset and minimizes some loss function $L$.

Suppose that the test unit $j$ is run for $m'_{j}$ time-steps and
$m = \sum_{j=1}^{M} m'_{j}$. Suppose that $\Delta^{(j)}$ is the difference
between the true RUL and the predicted RUL at time-step $j$, i.e.,
$\Delta^{(j)} = y^{(j)} - \hat{y}^{(j)}$. We have used two functions for
evaluating the models, the root-mean-square error (RMSE) and NASA's scoring
functions $\nasa$. These two functions are defined as follows.
\[ \rmse = \sqrt{\frac{1}{m}\sum_{j=1}^{m}(\Delta^{(j)})^2} \]
\[ \nasa = \sum_{j=1}^{m}\alpha_{j}\exp(|\Delta^{(j)}|) \mbox{ where, } \alpha_{j} = 
\left\{ \begin{array}{ll}
\frac{1}{13} & \mbox{ if } \Delta^{(j)} \geq 0 \\
\frac{1}{10} & \mbox{ if } \Delta^{(j)} < 0 \end{array} \right.\]
The coefficient $\alpha_{j}$ in NASA's scoring function is designed to penalize
over-estimation more than under-estimation.

%-------------------------------------------------------------------------------

% \pagebreak


%-------------------------------------------------------------------------------

\section{LSTM}

Recurrent Neural Networks (\textit{RNN}), perform exceptionally well at
prediction tasks involving temporal sequences. The dataset under consideration
consists of time series data of sensor readings. That is why, we decided to
investigate the potential of RNN networks for predicting the RUL. In our
experiments, we used LSTM networks instead of RNN.



\section*{References}

\medskip
\small
[1] Kwok L. Tsui, Nan Chen, Qiang Zhou, Yizhen Hai, Wenbin Wang, "Prognostics and Health Management: A Review on Data Driven Approaches", Mathematical Problems in Engineering, vol. 2015, Article ID 793161, 17 pages, 2015. https://doi.org/10.1155/2015/793161

[2] Biggio Luca, Kastanis Iason, "Prognostics and Health Management of Industrial Assets: Current Progress and Road Ahead", Frontiers in Artificial Intelligence,VOL 3, 2020, 88 pages, https://www.frontiersin.org/article/10.3389/frai.2020.578613, 10.3389/frai.2020.578613, 2624-8212

[3] M. Chao, C.Kulkarni, K. Goebel and O. Fink (2021). "Aircraft Engine Run-to-Failure Dataset under real flight conditions", NASA Ames Prognostics Data Repository (http://ti.arc.nasa.gov/project/prognostic-data-repository), NASA Ames Research Center, Moffett Field, CA

[4] Chao, Manuel Arias et al. “Fusing Physics-based and Deep Learning Models for Prognostics.” ArXiv abs/2003.00732 (2020)

[5] Shuochao Yao and Shaohan Hu and Yiran Zhao and Aston Zhang and Tarek Abdelzaher, "DeepSense: A Unified Deep Learning Framework for Time-Series Mobile Sensing Data Processing", ArXiv abs/1611.01942


%-------------------------------------------------------------------------------

% \begin{thebibliography}{9}


% \bibitem{phmdatadriven}
% Kwok L. Tsui, Nan Chen, Qiang Zhou, Yizhen Hai, Wenbin Wang.
% \\\textit{Prognostics and Health Management: A Review on Data Driven Approaches.}
% \\Mathematical Problems in Engineering, vol. 2015, Article ID 793161, 17 pages, 2015.
% \\\url{https://doi.org/10.1155/2015/793161}


% \bibitem{phmsurvey}
% Biggio Luca, Kastanis Iason.
% \\\textit{Prognostics and Health Management of Industrial Assets: Current Progress and Road Ahead.}
% \\Frontiers in Artificial Intelligence, Vol. 3, 2020, 88 pages
% \\\url{https://www.frontiersin.org/article/10.3389/frai.2020.578613}


% \bibitem{datadesc}
% M. Chao, C.Kulkarni, K. Goebel and O. Fink.
% \\\textit{Aircraft Engine Run-to-Failure Dataset under real flight conditions.}
% \\NASA Ames Prognostics Data Repository, NASA Ames Research Center, Moffett Field, CA.
% \\\url{http://ti.arc.nasa.gov/project/prognostic-data-repository}


% \bibitem{physicsdl}
% Chao, Manuel Arias et al.
% \\\textit{Fusing Physics-based and Deep Learning Models for Prognostics.}
% \\\url{https://arxiv.org/abs/2003.00732}


% \bibitem{deepsense}
% Shuochao Yao, Shaohan Hu, Yiran Zhao, Aston Zhang, Tarek Abdelzaher.
% \textit{DeepSense: A Unified Deep Learning Framework for Time-Series Mobile Sensing Data Processing.}
% \\\url{https://arxiv.org/abs/1611.01942}


% \end{thebibliography}

%-------------------------------------------------------------------------------


%-------------------------------------------------------------------------------
\end{document}
